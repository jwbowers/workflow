\documentclass[12pt]{article}
\usepackage{fancyvrb,natbib,url,array,dcolumn,float,booktabs,listings}
\usepackage[colorlinks=TRUE]{hyperref}
\usepackage{microtype}

\title{Six steps to a better relationship with your future self.}

\author{Jake Bowers\thanks{I owe many thanks to Mark Fredrickson,
    Brian Gaines, Kieran Healy, Kevin Quinn and Cara
    Wong for direct help on this document and to Mika LaVaque-Manty
    and Ben Hansen for many useful discussions on this topic. The
    source code for this document may be freely downloaded and
    modified from \url{https://github.com/jwbowers/workflow}.}}

\date{\today}

\newcolumntype{.}{D{.}{.}{1.2}}

\usepackage[noae,nogin]{Sweave} %Declaring this explicitly so that I can modify the Sweave environment here rather in Sweave.sty which will be c
\DefineVerbatimEnvironment{Sinput}{Verbatim} {xleftmargin=.5em,fontsize=\footnotesize}
\DefineVerbatimEnvironment{Soutput}{Verbatim}{xleftmargin=.5em,fontsize=\footnotesize}
\DefineVerbatimEnvironment{Scode}{Verbatim}  {xleftmargin=.5em,fontsize=\footnotesize}
\fvset{listparameters={\setlength{\topsep}{0pt}}}
\renewenvironment{Schunk}{\vspace{\topsep}}{\vspace{\topsep}}
  %Show comments
\fvset{fontsize=\footnotesize} %slightly smaller verbatim font

%% These next lines tell latex that it is ok to have a single graphic
%% taking up most of a page, and they also decrease the space arou
%% figures and tables.
\renewcommand\floatpagefraction{.9}
\renewcommand\topfraction{.9}
\renewcommand\bottomfraction{.9}
\renewcommand\textfraction{.1}
\setcounter{totalnumber}{50}
\setcounter{topnumber}{50}
\setcounter{bottomnumber}{50}
\setlength{\intextsep}{2ex}
\setlength{\floatsep}{2ex}
\setlength{\textfloatsep}{2ex}


\begin{document}
\frenchspacing
\VerbatimFootnotes
\maketitle




\begin{verse}
Do I contradict myself? \\
Very well then I contradict myself, \\
(I am large, I contain multitudes.) \\
\citep{whitman1855song}
\end{verse}

An idea is born in a coffee shop, a seminar, a quiet walk. On this
gray day in 2011, the idea dispels February's doldrums. The student
rushes home, mind racing, the cold ignored.
% Now he knows why he went to graduate school.

This idea inspires a seminar paper in the spring. A conference paper
arises from the seminar paper in collaboration with another student in
2012. A dissertation chapter descends from the conference paper in
2013. Other dissertation chapters take up 2014. A submission to a
journal with the original co-author and a new collaborator happens in
2015. Revision and resubmission wait until 2017 while harried editors,
reviewers and authors strive to balance research, teaching, service,
and life. By now, the three lucky collaborators work as professors in
three different universities. In 2018 a child is born and a paper is
published. The United Nations takes an interest in the paper in 2019
and hosts a conference to discuss implications of the research. In
2020 a first year graduate student in a coffee shop has an idea that
challenges the results in the now famous paper. She presents her paper
at a conference in 2021. What would happen if the authors had
controlled for X? Or included information now available but missing in
2012?  Or chosen a different likelihood function?  Will the United
Nations (now eager to act based the paper) make a wrong move?

The first author convenes a three way video conference with the other
collaborators during his homeward commute after putting his flying car
in auto-drive mode.\footnote{One assumes that video chatting during
  manual driving of flying cars will have been outlawed in his state
  by 2020.} The group must go back to the analyses. Which ones?  The
ones from 2011?  Or 2012? Or 2018? Where are the files? The next day,
one member of the group who has kept some hard-drives around out of
nostalgia finds some of the files.\footnote{This is the same guy who
  still owns cassette tapes and compact discs.} Now
re-analyses should be easy. Right? The student, now professor, should
remember the reason for those bits of code (or at least should
remember which series of mouse clicks were used to produce the numbers
for that crucial table as it was done in 2011 \ldots or was it
2015?). Right? And, of course, the way Microsoft
Word/Stata/SPSS/R/LISREL understands files and the way that 
machines in 2021 read and write them is the same --- since Windows and
Mac OS X have always existed and will always continue to exist more or
less as they currently exist. Right? And the group knows exactly which
bit of code produced which table and which figure, right? And they
wrote their code following Nagler's Maxims \citep{nagler1995coding}
and King's Replication Standard \citep{king1995replication}, right?

If the collaborators find themselves say ``Wrong'' in
answer to the questions posed here then reproducing, updating, or
changing the original analyses will take a lot of time. If
reproduction is hard to do, then the reputations of the scholars will
suffer and, more importantly, world peace will have been delayed. This
essay provides some suggestions for practices that will make such
reproduction occur much more easily and quickly in the event that
famous papers require special scrutiny. Specifically, this piece aims
to amplify some of what we already ought to know
\citep{king1995replication,nagler1995coding}, and to add to some of
those ideas given current practices, platforms, and possibilities.

%%% This data frame will be used throughout the analyses.
%%% Putting it up top, so we can use it in the comments section.

\section{Data analysis is computer programming.}\label{sec:data-analys-comp}

All results (numbers, comparisons, tables, figures) should arise from
code, not from a series of mouse clicks or copying and pasting. If I wanted
to re-create the figure from 2011 but include a new variable or
specification, I should be able to do so with just a few edits to the
code rather than knowledge of how I (or you) used a pointing device in
your graphical user interface some years ago.

Using R \citep{R:2011}, for example, I might specify that the file
\Verb+fig1.pdf+ was produced by the following commands in a file
called \Verb+fig1.R+.\footnote{The command \Verb+please-plot+ and some
  other R functions used in this essay come from the
  \Verb+MayIPleaseDoStatistics+ package which emphasizes politeness in
  data analysis. Functions like \Verb+please-plot+ can be blocked and
  more polite versions such as \Verb+may-I-please-have-a-plot+ can be
  required using \Verb+options(politeness=99)+ }

\begin{Verbatim}[fontsize=\footnotesize]
thedata <- read.csv("Data/thedata-15-03-2011.csv") ## Read the data
pdf('fig1.pdf') ## begin writing to the pdf file
please-plot(outcome by explanatory using thedata. red lines please.)
please-add-a-line(using model1)
## Note to self: a quadratic term does not add to the substance
## model2<-please-fit(outcome by explanatory+explanatory^2 using thedata
## summary(abs(fitted(model1)-fitted(model2)))
dev.off() ## stop writing to the pdf file
\end{Verbatim}


Now, in the future if I wonder how ``that plot on page 10'' was
created, I will know: (1) ``that plot'' is from a file called
\Verb+fig1.pdf+ and (2) \Verb+fig1.pdf+ was created in
\Verb+fig1.R+. In a future where R still exists, changing the figure
will require quick edits of commands already written.  In a future
where R does not exist, I will at least be able to read the plain text
R commands and use them to write code in my new favorite statistical
computing language: R scripts are written in
\href{http://en.wikipedia.org/wiki/Plain_text}{plain text}, and plain
text is a format that will be around as long as computer programmers
write computer programs.\footnote{Since R is open
  source, I will also be able to download an old version of R, download an
  old-fashioned open-source operating system (like Ubuntu 10), and
  run the old-fashioned statistical computing environment in the
  old-fashioned operating system in a virtual machine on my
  new-fashioned actual machine.}

Moreover, realize that file names send messages to your future
self. Name your files with evocative and descriptive names. Your
collaborators are less likely to call you at midnight asking for help
if your files are named \Verb+inequality-and-protest-figures.R+ than
if your files are called \Verb+temp9+ or
\Verb+supercalifragilisticexpialidocious+. The extension
\Verb+.R+ tells us and the operating system that the file contains R
commands. This part of the filename enables us to quickly search our
antique hard drives for files containing R scripts.

\paragraph{Step \thesection} If we know the provenance of results, 
future  or current collaborators can quickly and easily reproduce
and thus change and improve upon the work.

\section{No data analyst is an island for long.}
Data analysis involves a long series of decisions. Each decision
requires justification.  Some decisions will be too small and
technical for inclusion in the published article itself. These need to
be documented in the code itself \citep{nagler1995coding}. Paragraphs
and citations in the publication will justify the most important
decisions. So, one must code to communicate with yourself and
others. There are two main ways to avoid forgetting the reasons you
did something with data: comment your code and tightly link your code
with your writing.\footnote{One can also try the R command
  \Verb+put-it-in-my-mind(reason,importance='high')+ to firmly place a
  reason for a decision into the mind of the analyst. I myself have
  not had much luck with this function.}

\subsection{Code to communicate: Comment your code.}
Comments --- unexecuted text inside of a script --- are a message to
collaborators (including your future self) and other consumers of your
work. In the above code chunk, I used comments to explain the lines to
readers unfamiliar with R and to remember that I had tried a different
specification but decided not to use it because adding the squared
term did not really change the substantive story arising from the
model.\footnote{R considers text marked with \Verb+#+ as a comment.}
Messages left for your future self (or near-future others) help
retrace and justify your decisions as the work moves from seminar
paper to conference paper to poster back to paper to dissertation and
onwards.


% Luckily, if you are
% using a literate programming practice (ex. Sweave or odfWeave), you
% can write paragraphs to surround your code as well as technical
% comments in the code itself. If you are not using a strictly literate
% programming practice then use whatever commenting protocol exists in
% your analysis language. R, for example, considers text marked with
% ``\#'' as a comment. Here for example I comment about a linear model
% run for a paper and also include comments about another model run for
% due diligence.

 
% <<reg1,echo=TRUE,results=verb>>=
% ##  Repeat the regression run for Table~\ref{tab:protest}
% ##  Do political rights predict protest reporting?
% lm1 <- lm(protac00 ~ gini04 + meanpr, data=good.df)
% ##  Q: Was it worthwhile controlling for political rights?
% lm2 <- lm(protac00~gini04,data=good.df)
% coef(lm2)[2]-coef(lm1)[2]
% ##  A: Yes (around 3/4 an act of protest difference)
% @

Notice one other benefit of coding for an audience: we learn by
teaching. By assuming that others will look at your code, you will be
more likely to write clearer code, or perhaps even to think more
deeply about what you are doing as you do it.

Comment liberally. Comments are discarded when R runs analysis or
\LaTeX~turns plain text into page images, so only those who dig into
your work will see them.

\subsection{Code to communicate: Literate programming.}

\begin{quote}
  Let us change our traditional attitude to the construction of
  programs: Instead of imagining that our main task is to instruct a
  computer what to do, let us concentrate rather on explaining to
  human beings what we want a computer to do.
  \citep[][p. 97]{knuth1984literate}
\end{quote}

Imagine you discover something new (or confirm something old). You
produce a nice little report on your work for use in discussions of
your working group or as a memo for a web or reviewer appendix. The
report itself is a pdf file or some other format which displays page
images to ease reading rather than to encourage reanalysis and
rewriting. Eventually pieces of that report (tables, graphs,
paragraphs) ought to show up in, or at least inform, the publishable
paper. Re-creating those analyses by pointing, clicking, copying, or
pasting would invite typing error and waste time. Re-creating your
arguments justifying your analysis decisions would also waste
time. More importantly, we and others want to know why we did what we
did. Such explanations may not be very clear if we have some pages of
printed code in one hand and a manuscript in the other. Keep in mind
the distinction between the ``source code'' of a document (i.e. what
computation was required to produce it) and the visible, type-set page
image. Page images are great for reading, but not great for
reproducing or collaborating. The source code of any document
exchanged by the group must be available and executable.

How might one avoid these problems?
\href{http://en.wikipedia.org/wiki/Literate_programming}{Literate
  programming} is the practice of weaving code into a document ---
paragraphs, equations, and diagrams can explain the code, and the code
can produce numbers, figures, and tables (and diagrams and even equations and
paragraphs). Literate programming is not merely fancy commenting but
is about enabling the practice of programming itself to facilitate easy
reproduction and communication.

For example, in \S\ref{sec:data-analys-comp}, I suggested that we know
where ``that plot on page 10'' comes from by making sure we had a
\Verb+fig1.pdf+ file produced from a clearly commented plain text file
called something like \Verb+fig1.R+.  An even easier solution would be to
directly include a chunk of code to produce the figure inside of the
paper itself. This paper, for example, was written in plain text using
\LaTeX~ markup with R code chunks to make things like
Figure~\ref{fig:giniprot}. This combination of \LaTeX and R is called
Sweave \citep{Leis:2005}.\footnote{Support for Sweave is included with
  R.}

\begin{Verbatim}[fontsize=\footnotesize]
This paper, for example, was written in plain text using \LaTeX
markup with R code chunks to make things like
Figure~\ref{fig:giniprot}. This combination of \LaTeX and R is
called Sweave \citep{Leis:2005}.\footnote{Support for Sweave is
  included with R.}
\begin{figure}[h]
\centering
 <<fig1plot,fig=TRUE>>=
 ##  Make a scatterplot of Protest by Inequality
 with(good.df,plot(gini04,protac00,xlab='Gini Coefficient 2004 (UNDP)',
                   ylab='Mean Protest Activities\n(World Values Survey 1980-2000)'))
 ##  Label a few interesting points
 with(good.df[c("EGY","JOR","USA","SWE","CHL"),],
      text(gini04,protac00,labels=Nation))
 @
\caption{Protest activity by income inequality \citep[from][]{norris2009data}.}
\label{fig:giniprot}
\end{figure}
\end{Verbatim}


\begin{figure}[H]
  \begin{center}
\includegraphics{workflow-fig1plot}
    \caption{Protest activity by income inequality \citep[from][]{norris2009data}.}
    \label{fig:giniprot}
  \end{center}
\end{figure}


By using \Verb+\label{fig:giniprot}+, I do not need to keep track of
the figure number, nor do extra work when I reorganize the document in
response to reviewer suggestions. Nor do I need a separate
\Verb+fig1.R+ file or \Verb+fig1.pdf+ file.  Tables and other
numerical results are also possible to generate within the source
code of a scholarly paper. Those who view the code for this essay will
see how Table~\ref{tab:protest} was also generated directly from a
regression object.\footnote{\citet{beck2010reg} inspired this
  particular presentation of a linear model.}



% latex table generated in R 2.12.2 by xtable 1.5-6 package
% Sat Mar 26 09:57:24 2011
\begin{table}[H]
\begin{center}
\begin{tabular}{l....}
   \toprule 
 &
             \multicolumn{1}{c}{Coef} &
             \multicolumn{1}{c}{Std. Err.} &
             \multicolumn{2}{c}{95\% CI} \\ \cmidrule(r){2-5} Intercept & 1.5 & 0.2 & 1.2 & 1.8 \\ 
  Income Inequality (lower=more equal) & -1.0 & 0.4 & -1.9 & -0.2 \\ 
  Mean Political Rights (lower=more rights) & -0.2 & 0.0 & -0.2 & -0.1 \\ 
   & \multicolumn{4}{c}{n: 53,   resid.sd: 0.28, R$^2$: 0.57} \\ \bottomrule\end{tabular}
\caption{People living in countries with unequal
  income distributions report less protest activity to World Values
  Survey interviewers than people living in countries with relatively
  more equal income distributions, adjusting for average political
  rights as measured by Freedom House 1980--2000. Data from \citep{norris2009data}.}
\label{tab:protest}
\end{center}
\end{table}



Literate data analysis is not the same as Sweave, even if Sweave is a
nice implementation.\footnote{The R project has a task view devoted to
  \href{http://cran.r-project.org/web/views/ReproducibleResearch.html}{reproducible
    research} listing many of the different approaches to literate
  programming for R.}  \href{http://www.lyx.org/}{LyX} offers a
WYSIWYG environment for \LaTeX~ that supports Sweave. And the
\Verb+odfWeave+ package in R allows the use of OpenOffice documents in
exactly the same way.\footnote{A quick Google search of ``Sweave for
  Stata'' turned up lots of resources for literate programming with
  Stata.} If your workflow does not involve \LaTeX~ and R, you can
still implement some of the principles here. Imagine creating a style
in Microsoft Word called ``code'' which hides your code when you print
your document, but which allows you do at least run each code chunk
piece by piece [or perhaps there are ways to extract all text of style
``code'' from a Microsoft Word document]. Or imagine just using some
other kind of indication linking paragraphs to specific places in code
files. There are many ways that creative people can program in a
literate way. 

Literate programming need not go against the principle of modular data
analysis \citep{nagler1995coding}. In my own work I routinely have
several different Sweave files that fulfill different functions, some
of them create \LaTeX code that I can \Verb+\input+ into my
\Verb+main.tex+ file, others setup the data, run simulations, or allow
me to record my journeys down blind alleys. Of course, when we have
flying cars running on autopilot, perhaps something other than
Sweave will make our lives even easier. Then we'll change.

\paragraph{Step \thesection}

We analyze data in order to explain something about the world to other
scholars and policy makers. If we focus on explaining how we got our
computers to do data analysis to human beings, we will do a better job
with the data analysis itself: we will learn as we focus on teaching,
and we will avoid errors and save time as we ensure that others
(including our future selves) can retrace our steps. A document than
can be ``run'' to reproduce all of the analyses also instills
confidence in readers and can more effectively spur discussion and
learning and cumulation of research.

\section{Meaningful code requires data.}

All files containing commands operating on data must refer to a data
file. A reference to a data file is a line of code the analysis
program will use to operate on (``load''/ ``open'' / ``get'' /
``use'') the data file. One should not have to edit this line on
different computers or platforms in order to execute this command.
Using R, for example, all analysis files should have
\Verb+load('thedata.rda')+ or \Verb+read.csv('thedata.csv')+ or some
equivalent line in them, and \Verb+thedata.csv+ should be stored in
some place easy to find (like in the same directory as the file or
perhaps in \Verb+'Data/thedata.rda'+). Of course, it never hurts to
drop in a comment pointing to the data file.

Where should one store data files? Anobvious solution is always to
make sure that the data file used by a command file is in the same
directory as the command file. More elegant solutions require all
co-authors to have the same directory structure so that
\Verb+load('Data/thedata.rda')+ means the same thing on all computers
used to work on the project. This kind of solution is one of the
things that formal version control systems do well (as discussed
a bit more in \S\ref{sec:vers-contr-prev}).

The principle of modularity suggests that you separate data cleaning,
processing, recoding, and merging from analysis in different files
\citep{nagler1995coding}. So, perhaps your analysis oriented files
will \Verb+load('cleandata.rda')+ and a comment in the code will alert
the future you (among others) that \Verb+cleandata.rda+ was created
from \Verb+create-cleandata.R+ which in turn begins with
\Verb+read.csv(\url('http://www.greatfreedata.gov/dirtydata.csv'))+. Such
a data processing file will typically end with something like
\Verb+save('cleandata.rda')+ so that we are doubly certain about the
provenance of the data.\footnote{Of course, if you need math or
  paragraphs to explain what is happening in these files, you might
  prefer to make them into Sweave files, for which the conventional
  extension is .Rnw. So you'd have \Verb+create-cleandata.Rnw+ which
  might explain and explore the different coding decisions you made,
  perhaps involving a factor analysis and diagnostic plots.}

Now, if in the future we wonder where \Verb+cleandata.rda+ came from, we
might search for occurrences of `cleandata' in the files on our
system. However, if such searching among files is a burden, an even
nicer solution is to maintain a file for each project called
``MANIFEST.txt'' or ``INDEX.txt'' or ``README.txt'' which lists the
data and command files with brief descriptions of their functions and
relations.

\paragraph{Step \thesection} We should know data where the data came
from and what operations were performed on which set of data.

In the good old days, when we executed our LISREL code in batch mode,
we had no choice but to tell the machine clearly, in a few easy to
understand and informative lines, what files (with filenames no longer
than 8 characters) to use:

\begin{Verbatim}[fontsize=\footnotesize]
DA NI=19 NO=199 MA=CM
LA=basic.lab
CM FI=basic.cov
\end{Verbatim}

The fact that I need to articulate this idea at all arises because of
graphical user interfaces: it is all too easy to use the mouse to load
a data file into memory and then to write a script to analyze this
file without ever noting the actual name or location of the data file.

\section{Version control prevents clobbering and reconciles history.}\label{sec:vers-contr-prev}

Group work requires version control.\footnote{
  \citet{fredrickson2011tpm} and \citet{healy2011tpm} in this issue
  also explain what version control is and why we might want to use
  it.} Many people are familiar with the ``track changes'' feature in
modern WYSIWYG word processors or the fact that Dropbox allows one to
recover previous versions of files. These are both kinds of version
control. More generally, when we collaborate, we'd like to do a variety
of actions with our shared files. Collaboration on data analytic
projects is more productive and better when (1) it is easy to see what
has changed between versions of files; (2) members of the team feel
free to experiment and then to dump parts of the experiment in favor
of previous work while merging the successful parts of the experiment
into the main body of the paper; (3) the team can produce have
``releases'' of the same document (one to MPSA, one to APSR,
one to your parents) without spawning many possibly conflicting copies
of the same document; (4) people can work on the same files at
the same time without conflicting with one another (and can reconcile
their changes without too much confusion and clobbering).  Clobbering
is what happens when your future self or your current collaborator
saves an old version of a file over a new version, erasing good
work by accident.

Of course if you rely on Dropbox or ``track changes'' for version
control, you must communicate with other folks in your group before
you edit existing files. Only one of you can edit and save a given
file at a time. This prevents your work (or your colleagues work) from
getting lost when you both try to save the same file on top of each
other. If you find that you need to work on the same files at the same
time, then you should work on establishing your own shared version
control system. Free options include launchpad, github, sourceforge
for open source projects (i.e. papers you are writing which you are
happy to share with others as you write). Each of those services
include paid versions too. One may also use Dropbox as a kind of
server for version control: for example, one may copy files from the
Dropbox directory into a local working directory so as to avoid
clobbering and then work on merging changes by hand before copying
back to the Dropbox directory and replacing existing files.

We use subversion with our own research group, and I use it for all of
my own projects (except this one, for which I am experimenting with
git). Subversion and bazaar and git are all great. They mainly differ
in the extent to which you need to run a server. Subversion requires a
server.\footnote{If you already pay to host a website, you may already
  have the right to run a subversion or git server there. Your
  university or institute may have a version control system running
  somewhere on campus. And Google will direct you to many helpful
  people who have installed such servers on their own diverse desktop
  machines. Github requires that you pay to host private
  repositories.}

Of course, one may take advantage of many of the benefits of formal
version control systems with some disciplined systems for file and
directory organization.  An excellent, simple, and robust version
control system is to rename your files with the date and time of
saving them: thedoc.tex becomes thedoc25-12-2011-23:50.tex.  Be sure
to include year in the file names --- remember, the life of an idea is
measured in years. If you are wise enough to have saved your documents
as plain text then you can easily compare documents using the many
utilities available for comparing text files.\footnote{Adobe Acrobat
  allows one to compare differences in pdf files. OpenOffice supports
  a ``Compare Documents'' option. And Google Docs will report on the
  version history of a document.} When you reach certain milestones
you can rename the file accordingly: thedocAPSA2009.tex --- for the
one sent to discussants at APSA --- or thedocAPSR2015.tex --- for the
version eventually sent to the APSR six years after you presented it
at APSA. The formal version control systems I mentioned above all
allow this kind of thing and are much more elegant and capable, but
you can do it by hand too as long as you don't mind taking up a lot of
disk space and having many ``thedoc...''  files around. If you do
version control by hand, spend a little extra time to ensure that you
do not clobber files when you make mistakes typing in the
file-names. And, if you find yourself spending extra time reconciling
changes made by different collaborators by hand, remember this is a
task that modern version control systems take care of quickly and
easily.


\paragraph{Step \thesection} Writing is rewriting. Thus, all writing
involves versions. When we collaborate with ourselves and others we
want to avoid clobbering and we want to enable graceful reconciliation
of rewriting. One can do these things with formal systems of software
(like subversion or git or bazaar) or with formal systems of file
naming, file comparing, and communication or, even better, with
both. In either case, plain text files will make such tasks easier,
will take up less disk space, and be easier to read for the future
you.


% This is so true that the
% source of the quote is unclear.\footnote{Could it be Paul Abbott?
%   \url{http://www.bbc.co.uk/writersroom/insight/paul_abbott.shtml}}
% Here is a quote with a known source: ``Writing and rewriting are a constant search for what it is one is
% saying'' (John Updike but ??Writing With Style: Conversations on the Art of Writing, 1975 
% by John Trimble ??)


\section{Minimize error by testing.}

Now, back to that famous article of 2018. After reading the conference
paper critique of 2021 (that came from the seminar paper of 2020), the
statisticians at the UN worry about the bootstrap confidence intervals
presented in the original paper.\footnote{Perhaps they should be
  worried about the deeper substantive critiques offered by the
  student, but they are statisticians and so focus on the stats. The
  policy makers of 2021 were cowed by the methodological virtuosity of
  the 2018 article, and so, even though they had the same substantive
  concerns as the student, they kept their mouths shut at the
  mini-conference to avoid looking dumb in front of their bosses.} So,
now the authors would like to evaluate their bootstrap procedure. Although
nice code exists for bootstrapping linear models, no nice code exists
to bootstrap the bootstrap. Of course, the code required is not
complex, but since they are writing custom code they worry about
getting it right. As they've struggled to respond to the critiques of
their paper, they've had lots of time to appreciate problems arising
from bugs, errors, and typos in data analysis and code.

Now, if they had a moment to think in between teaching that new class,
reading books for an awards committee, evaluating application files
for the admissions committee, feeding popsicles to a sick child, and
undertaking the odd bit of research, they might say to themselves,
``Before I write new code, I should write a test of the code. I should
write a little bit of code that lets me know that my double-bootstrap
procedure actually does what it is supposed to do.''

Of course, this idea, like most others, is not new. The desire to
avoid error looms large when large groups of programmers write code
for multi-million dollar programs. The idea of
\href{http://en.wikipedia.org/wiki/Test-driven_development}{test
  driven development} and the idea that one ought to create tests of
\href{http://en.wikipedia.org/wiki/Unit_testing}{small parts of one's
  code} arose to address such concerns. For the social scientist
collaborating with her future self and/or a small group of
collaborators, here is an example of this idea in a very simple form:
Say I want to write a function to multiply a number by 2. If my
function works, when I give it the number 4, I should see it return
the number 8 and when I give it -4, I should get -8.

\begin{Schunk}
\begin{Sinput}
 ##  The test function:
 test.times.2.fn <- function(){
   ##  This function tests times.2.fn
   if (times.2.fn(thenumber=4) ==  8 &
       times.2.fn(thenumber=-4) == -8) { 
     print("It works!")
   } else { print("It does not work!")
          }
 }
 ##  The function:
 times.2.fn <- function(thenumber){
   ##  This function multiplies a scalar number by 2
   ##  thenumber is a scalar number
   thenumber+2
 }
 ##  Use the test function
 test.times.2.fn()
\end{Sinput}
\begin{Soutput}
[1] "It does not work!"
\end{Soutput}
\end{Schunk}

Ack! I mistyped ``+'' for ``*''. Good thing I wrote the
test!\footnote{A more common example of this kind of testing occur
  everyday when we recode variables into new forms but look at a
  crosstab of the old vs. new variable before proceeding.}

\paragraph{Step \thesection} No one can forsee all of the ways that a
computer program can fail. One can, however, at least make sure that
it succeeds in doing the task motivating the writing of the code in
the first place.

\section{Copy and improve on others' examples.}

Lots of people are thinking about ``reproducible research'' and
``literate programming'' these days. Google those terms. Of course, the
devil is in the details: Here I list a few of my own attempts at enabling
reproducible research. You'll find many other inspiring examples on
the web. Luckily, the open source ethos aligns nicely with academic
incentives, so we are beginning to find more and more people offering
their files online for copying and improvement. By the way, if you do
copy and improve, it is polite to alert the person from whom you made
the copy about your work.

I have experimented with three systems so far: (1) for one paper we
simply included a Sweave document and data files into a compressed
archive \citep{bowers2005dataverse}; (2) for another more computing
intensive paper we assembled a set of files that enabled reproduction
of our results using the \texttt{make} system
\citep{bowers2008dataverse}; and (3) recently I have tried the
``compendium'' approach
\citep{gentleman2005reproducible,gentleman2007statistical} which
embeds an academic paper within the R package system
\citep{bowers2011dataverse}. The benefit of this last approach is that
one is not required to have access to a command line for \Verb+make+:
the compendium is downloadable from within R using
\Verb+install.packages()+ and is viewable using the \Verb+vignette()+
function in any operating system than runs R.\footnote{Notice that my
  reproduction archives and/or instructions for using them are hosted
  on the
  \href{http://thedata.org/book/learn-about-project}{Dataverse}, which
  is another system designed to enhance academic collaboration across
  time and space.}  The idea that one ought to be able to install and
run and use an academic paper just as one installs and uses
statistical software packages is very attractive, and I anticipate
that it will become ever easier to turn papers into R packages as
creative and energetic folks turn their attention to the question of
reproducible research.

\paragraph{Step \thesection} We all learn by doing. When we share
reproduction materials we improve both cumulation of knowledge and our
methods for doing social science
\citep{freese2007replication,king1995replication}.  As we copy and
improve upon each other's modes of doing work we enhance our collective
ability to believe each other and for future scholars to believe us, too.

\section{Remember that research ought to be credible communication.}
\begin{quote}
  [I]f the empirical basis for an article or book cannot be
  reproduced, of what use to the discipline are its
  conclusions?  What purpose does an article like this serve?
  \cite[445]{king1995replication}
\end{quote}

We all always collaborate. Many of us collaborate with groups of
people at one moment in time as we race against a deadline. All of us
collaborate with ourselves over time.\footnote{What is a reasonable
  time-span for which to plan for self-collaboration on a single idea?
  Ask your advisers how long it took them to move from idea to dissertation to
  publication.}  The time-frames over which collaboration are required
--- whether among a group of people working together or within a
single scholar's productive life or probably both --- are much longer
than any given version of any given software will easily exist. Plain
text is the exception. Thus, even as we extol version control systems,
one must have a way to ensure future access to them in a form that
will still be around when sentient cockroaches finally join political
science departments (by then dominated by cetaceans after humans are
mostly uploads).\footnote{The arrival of the six-legged social
  scientists revives Emacs and finally makes Ctrl-a Ctrl-x Esc-x
  Ctrl-c a \href{http://kieran.healy.usesthis.com/}{reasonable key
    combination}.}

But what if the UN never hears of your work, or, by some cruel fate, that your
article does not spawn debate? Why then would you spend time to
communicate with your future self and others? My own answer to this
question is that I want my work to be credible and useful to myself
and other scholars even if each article does not immediately change
the world.  What I report in my data analyses should have two main
characteristics: (1) the findings of the work should not be a matter
of opinion; and (2) other people should be able to reproduce the findings. That is,
the work represents a shared
experience --- and an experience shared without respect to the
identities of others (although requiring some common technical
training and research resources).
% Such work should make us change how
% we act --- or at least, it ought to stand on stronger epistemological
% ground than other claims about experiences which ought to be shared or
% shareable.

Assume we want others to believe us when we say something. More
narrowly, assume we want other people to believe us when we say
something about data: ``data'' here can be words, numbers, musical
notes, images, ideas, etc \ldots The point is that we are making some
claims about patterns in some collection of stuff. Now, it might be
easy to convince others that ``this collection of stuff is different
from that collection of stuff'' if those people were looking over our
shoulders the whole time that we made decisions about collecting the
stuff and broke it up into understandable parts and reorganized and
summarized it. Unfortunately, we can't assume that people are willing
to shadow a researcher throughout her career. Rather, we do our work
alone or in small groups and want to convince other distant and future
people about our analyses.

Now, say your collections of stuff are large or complex and your
chosen tools of analyses are computer programs. How can we convince
people that what we did with some data with some program is credible,
not a matter of whim or opinion, and reproducible by others who didn't
shadow us as we wrote our papers? This essay has suggested a few
concrete ways to enhance the believability of such scholarly work. In
addition, these actions (as summarized in the section headings of this
essay) make collaboration within research groups more effective.
Believability comes in part from reproducibility and researchers
often need to be able to reproduce in part or in whole what different
people in the group have done or what they, themselves, did in the past.

In the end, following these practices and those recommended by
\citet{fredrickson2011tpm} and \citet{healy2011tpm} in this issue
allows your computerized analyses of your collections of stuff to be
credible.  Finally, if the UN quibbles with your analyses, your future
self can shoot them the archive required to reproduce your
work.\footnote{Since you used plain text, the files will still be
  intelligible, analyzed using commented code so that folks can
  translate to whatever system succeeds R, or since you used R, you
  can include a copy of R and all of the R packages you used in your
  final analyses in 2018 in the archive itself. You can even throw in
  a copy of Ubuntu 10 and an open source virtual machine running the
  whole environment.} You can say, ``Here is everything you need to
reproduce my work." To be extra helpful you can add ``Read the README
file for further instructions." And then you can get on with your life:
maybe the next great idea will occur when your 4-year-old asks a wacky
question after stripping and painting her overly cooperative
1-year-old brother purple, or teaching a class, or in a coffee shop,
or on a quiet walk.

\bibliographystyle{apsr}
\bibliography{/Users/jwbowers/Documents/BIB/trunk/big}



\end{document}
\begin{Schunk}
\begin{Sinput}
 options(prompt="> ",continue="+ ")
\end{Sinput}
\end{Schunk}
